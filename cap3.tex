\chapter{Protocolo de comunicación}

% Que es el protocolo de comunicación
El protocolo de comunicación consiste en una selección de mensajes estructurados, diseñados para realizar ciertas acciones sobre una simulación de vuelo, mensajes que son enviados desde un equipo externo como un microcontrolador u otro computador convencional, a otro equipo que esté ejecutando software de simulación.

% Para que sirve
Este protocolo ofrece la capacidad de trabajar de manera programática con las variables que gobiernan una simulación, por ejemplo puede hacer todo lo que ya incluye X-Plane en su interfaz de usuario como guardar datos a un Excel o cargar un estado específico, pero además permite acceder a variables internas de la simulación y modificarlas arbitrariamente en tiempo real.

% Ejemplos de uso
Ejemplos de uso pueden ser ejecutar acciones sencillas como automatizar una misión, o analizar como afectan diversas condiciones externas como el clima o comandos del piloto a una maniobra de manera iterativa, reiniciando repetidamente el estado de la aeronave a uno inicial pero alterando ciertas variables. Más interesante es la nueva habilidad de poder responder en tiempo real al estado de la simulación, que permite desarrollar sistemas de control desde simples PID en Python o C hasta complejos modelos en Simulink.

% En la U servira para practicar programación en Python y proyectos
En el contexto del laboratorio de técnicas aeroespaciales, el diseño del protocolo de comunicación busca alcanzar un balance entre versatilidad y facilidad de uso, que permitirá a estudiantes interesados aplicar sus conocimientos de Python y poder manipular la simulación mediante código comprensible, pero que también sirva de herramienta potente para la investigación y no sea un factor limitante debido a que no posea cierta funcionalidad que se requiera.

\section{Microprocesadores}

\subsection{Canales de comunicación}

\subsection{Lenguajes de programación}

\section{Interfaz a simulador}

\subsection{Flight Simulator Universal Inter-Process Communication}

\section{Establecimiento del protocolo}

\subsection{Alcance}

\subsection{Decisiones de diseño}

\subsubsection{Servidor}

\subsubsection{Cliente}

\subsection{Protocolo}



